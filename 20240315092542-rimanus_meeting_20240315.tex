% Created 2024-03-15 Fri 09:58
% Intended LaTeX compiler: pdflatex
\documentclass[11pt]{article}
\usepackage[utf8]{inputenc}
\usepackage[T1]{fontenc}
\usepackage{graphicx}
\usepackage{longtable}
\usepackage{wrapfig}
\usepackage{rotating}
\usepackage[normalem]{ulem}
\usepackage{amsmath}
\usepackage{amssymb}
\usepackage{capt-of}
\usepackage{hyperref}
\date{\textit{<2024-03-15 Fri>}}
\title{RIMANUS Meeting 20240315 20240315092542-rimanus\textsubscript{meeting}\textsubscript{20240315}}
\hypersetup{
 pdfauthor={},
 pdftitle={RIMANUS Meeting 20240315 20240315092542-rimanus\textsubscript{meeting}\textsubscript{20240315}},
 pdfkeywords={},
 pdfsubject={},
 pdfcreator={Emacs 29.2 (Org mode 9.6.15)}, 
 pdflang={English}}
\begin{document}

\maketitle

\section{Define the Problem}
\label{sec:org80e9540}
\subsection{Question: Which scenario is our focus?}
\label{sec:orgb5958ee}
\subsection{Different scenarios have different flow velocities.}
\label{sec:orgd90b4d4}
\begin{itemize}
\item Normal condition: flow velocity is about 6 m/s.
\item \href{20240307154931-core_disruptive_accident_cda.org}{Core Disruptive Accident (CDA)}: there is likely no such high velocity.
\end{itemize}
\section{Scenario: CDA}
\label{sec:orgff8161f}
\subsection{Questions:}
\label{sec:orgfd1b176}
\begin{itemize}
\item What is the process of CDA?
\item Which process would be interesting for other researchers and significant for the nuclear safety?
\item Or which process is particularly blank and demanded?
\end{itemize}
\subsection{Literature review:}
\label{sec:orge8fd342}
\begin{itemize}
\item There is natural circulation flow in SFR.
\item The sudden loss of coolant may never happen.
\item The reason for sodium boiling is generally "Breakdown of the forced circulation".
\item It takes longer before the sodium boiling finally happens.
\end{itemize}
\section{Bubble behavior}
\label{sec:org679b83b}
\begin{itemize}
\item There is not enough study on bubble behaviors on liquid metal.
\item Then the question comes: How similar are water and liquid sodium? What is the difference between them?
\item Difference:
\begin{itemize}
\item Heat transfer: Liquid sodium >> Water
\item Viscosity: Water > Liquid sodium
\item Surface Tension: Liquid sodium > Water
\end{itemize}
\item Coalescence:
\begin{itemize}
\item (Water) Collision \(\uparrow\) Coalescence \(\uparrow\)
\item (Water) \href{20240313140039-viscosity.org}{Viscosity} \(\uparrow\) Coalescence \(\uparrow\), \href{20240312202630-surface_tension.org}{Surface Tension} \(\uparrow\) Coalescence \(\downarrow\)
\item (Water) \href{20240312202630-surface_tension.org}{Surface Tension} \(\uparrow\) Coalescence \(\downarrow\)
\end{itemize}
\item Inference:
\begin{itemize}
\item It is harder for bubbles in liquid sodium to coalescence.
\end{itemize}
\end{itemize}
\section{Quotes}
\label{sec:org9f52ab6}
\begin{quote}
Together with a high capacity of natural circulation flow establishment in SFR, the boiling of the total sodium mass under residual power deposition (after loss of forced circulation) takes several hours. This duration is about 15 hours to get the total coolant boiling, without taking into account natural circulation flow. (\href{20240315093632-a_comparative_study_on_severe_accident_phenomena_related_to_melt_progression_in_sfr_and_pwr.org}{bachrataComparativeStudySevere2021})
\end{quote}

\begin{quote}
In SFR ‘pool-type’ reactors, the loss of coolant due to loss of pressurization e.g. pipe rupture is practically eliminated. Consequently, the severe accident scenarios with fast total loss of cooling are practically eliminated. (\href{20240315093632-a_comparative_study_on_severe_accident_phenomena_related_to_melt_progression_in_sfr_and_pwr.org}{bachrataComparativeStudySevere2021})
\end{quote}
\end{document}
